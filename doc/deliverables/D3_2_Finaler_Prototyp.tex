\input{header}

\title{MyThOS D3.2 Finaler Prototyp}
\author{Stefan Bonfert, Vladimir Nikolov, Robert Kuban, Randolf Rotta}

\hypersetup{
  pdftitle={MyThOS D3.2 Finaler Prototyp},
  pdfsubject={MyThOS Deliverable},
  pdfauthor={Stefan Bonfert, Vladimir Nikolov, Robert Kuban, Randolf Rotta},
  pdfkeywords={MyThOS} % comma separated
}

\begin{document}
\selectlanguage{ngerman}
\maketitle

\begin{abstract}
Der finale Prototyp ist das primäre Softwaredeliverable des Arbeitspaketes und integriert die einzelnen Softwareergebnisse und bildet auch die Basis für die folgende Evaluierung. Eine lauffähige Version des Kernels wurde projektintern für Benchmarks bereitgestellt. Diese ist aber äußerst fragil und daher nicht wirklich für Außenstehende benutzbar. Die BTU hat den Betriebssystemkernel seitdem kräftig überarbeitet und will eine stabile Version für Dritte und Folgeprojekte veröffentlichen.
\end{abstract}

\newpage
\tableofcontents
% --- content ------------------------------------------------------------------

\section{Übersicht}


% ------------------------------------------------------------------------------
% \bibliographystyle{alpha}
% \bibliography{literature}

\end{document}

\input{header}

\title{MyThOS D4.2.2 Developer Tools}
\author{Stefan Bonfert, Vladimir Nikolov, Robert Kuban, Randolf Rotta}

\hypersetup{
  pdftitle={MyThOS D4.2.2 Developer Tools},
  pdfsubject={MyThOS Deliverable},
  pdfauthor={Stefan Bonfert, Vladimir Nikolov, Robert Kuban, Randolf Rotta},
  pdfkeywords={MyThOS} % comma separated
}

\begin{document}
\selectlanguage{ngerman}
\maketitle

\begin{abstract}

Zur Unterstützung der Entwickler werden eine Reihe von
Softwareentwicklungswerkzeugen zur Verfügung gestellt. Dabei wird soweit möglich
auf bestehende Werkzeuge zurückgegriffen und durch Anpassung und Erweiterungen
eine Umsetzung realisiert.

\end{abstract}

\newpage
\tableofcontents
\newpage
% --- content ------------------------------------------------------------------

\section{Deliverable D4.2.2 und das Gesamtpaket D.4.2 - \emph{Developer Tools}}

Dieses Dokument erweitert den aktuellen Stand der Softwareentwicklungswerkzeuge,
welcher zuvor im Deliverable D4.2.1 erfasst wurde.
Diese Werkzeuge werden bei der Entwicklung des Kernels und bei der Entwicklung
bzw. Anpassung der Anwendungen an die neue Ausführungsumgebung verwendet und
werden im Gesamtpaket D4.2 zusammengefasst.

Das Gesamtpaket D4.2 dient somit der Erstellung eines minimalen Werkzeugkastens,
welcher für die Entwicklung von Anwendungen und deren Betrieb bzw. Evaluation
mit MyThOS bereitgestellt wird. Die verwendeten Werkzeuge sind größtenteils frei
verfügbar (z.B. open source). Proprietäre Produkte, wie der Intel C/C++
Compiler, die z.B. für eine Performance-Steigerung der Anwendungen auf der
Evaluationshardware verwendet werden können, werden ebenfalls diskutiert, obwohl
diese nicht zwingendermaßen für die Entwicklung notwendig sind (vgl. D4.2.1).

Darüber hinaus werden in D4.2 auch Vorgänge und Tools beschrieben, welche bei
der Auswertung der Anwendungsperformance verwendet werden und auch solche, die
beim aktuellen Betrieb der Anwendungen für das Erstellen und für die Auswertung
von Simulationstests benötigt werden. Letztere werden auch für die Evaluation
der Anwendungsperformance mit der neuen Betriebssystem-Umgebung zum Einsatz
kommen.

In diesem Dokument sind lediglich diejenigen Werkzeuge und Vorgänge beschrieben,
die seit der Veröffentlichung von Deliverable D4.2.1 verändert oder zusätzlich in den
Entwicklungsprozess aufgenommen wurden. Somit ist dieses Dokument als
Erweiterung von D4.2.1 zu betrachten und schließt das Gesamtpaket D4.2 ab.

\section{Konfigurations- und Build-Tools}

\mythos unterstützt einen modularen und konfigurierbaren Kernel sowie
entsprechende Werkzeuge, mit denen das Erstellen eines für den jeweiligen
Anwendungsfall und Zweck angepassten Kernel-Abbilds automatisiert ablaufen kann.
Das zentrale Werkzeug für die Verwaltung der Kernel-Konfiguration ist das
python-basierte sog. \texttt{mcconf}-Tool, welches im Verzeichnis
\texttt{\{mythos\}/3rdparty/mcconf} zu finden ist. Die Hauptaufgabe
dieses Werkzeugs ist es eine Ansammlung von zu übersetzenden Quellcode-Dateien
zu erstellen, die eine gewünschte bzw. benutzer-definierte Kernel-Konfiguration
umsetzt, und auch Abhängigkeiten auf der Code-Ebene zu verwalten. Damit ist
es ebenfalls möglich, in einem Schritt automatisiert mehrere Kernel-Versionen
für verschiedene Anwendugsfälle und Hardware-Platformen zu erzeugen.

In \mythos werden zusammengehörende Dateien, die eine eigenständige Einheit
bilden, zu Modulen zusammengefasst. Dadurch können sowohl Anforderungen von
Modulen wie besipielsweise Compiler-Unterstützung, als auch Unterschiede
zwischen den Kernel-Varianten für unterschiedliche Plattformen berücksichtigt
werden. Zudem erleichtert die Modul-orientierte Struktur die Evaluation
verschiedener Implementierungsvarianten, da einzelne Module einfach austauschbar
sind.

Jedes Modul stellt eine Menge von Dateien bereit und kann von anderen Dateien
und Modulen abhängen. Solche Abhängigkeiten können beispielsweise Header-Dateien
oder spezielle Implementierungsvarianten einzelner Objekte darstellen. Die
meisten Abhängigkeiten müssen nicht manuell spezifiziert werden, sondern werden
durch das \texttt{mcconf} Tool automatisiert aus den \texttt{\#include}
Anweisungen im Quellcode extrahiert. Die Definition eines jeden Kernel-Moduls
liegt als Datei (\texttt{mcconf.module}) im entsprechenden Verzeichnis im
Kernel-Quellcode vor. Diese spezifiziert die vom Modul bereitgestellten
Objekte und Dateien, benötigte Objekte und auch spezielle Konfigurationen und
Anweisungen für den Build-Prozess (Compiler und Linker). Das \texttt{mcconf}
Tool traversiert automatisch die Verzeichnisstruktur des Kernel-Quellcodes
und registriert alle auffindbaren Moduldefinitionen.

Um automatisiert alle für eine bestimmte Konfiguration benötigten Dateien
zusammenzustellen wird eine Beschreibung der gewünschten Kernel-Konfiguration
benötigt. Solche Konfigurationen können als Dateien mit definierter
Struktur und Syntax bereitgestellt und abgespeichert werden. Diese tragen die
Endung \texttt{.config}. Beispiele bereits vordefinierter Konfigurationen für
\texttt{amd64} (\texttt{kernel-amd64.config}) und \texttt{KNC}
(\texttt{kernel-knc.config}) Architekturen liegen im \mythos-Hauptverzeichnis
vor. Die entsprechenden Kernel-Konfigurationen werden durch das dort ebenfalls
bereitgestellte Makefile automatisch erzeugt und liegen dann
als Unterverzeichnisse und Dateien vor. Jede Spezifikation legt fest,
welche Module für eine bestimmte Ziel-Konfiguration benötigt werden, um ein
lauffähiges Kernel-Image zu erstellen. Zur Auflösung der hier
beschriebenen Abhängigkeiten erzeugt das \texttt{mcconf} Tool eine Liste aller
benötigten Dateien und Module. Anschließend wird für jedes benötigte Objekt ein
Modul ausgewählt und zum Kernel-Abbild hinzugefügt, falls dieses eines der
benötigten Objekte bereitstellt.

Falls zwei Module die selbe Datei bereitstellen entsteht ein Konflikt, der vom
Entwickler manuell aufgelöst werden muss, indem er eines der möglichen Module
auswählt. Zur einfacheren Auswahl größerer Mengen von Modulen können Tags
verwendet werden, die Teil der Modulbeschreibung sind. So sind beispielsweise
alle Module, die auf die Xeon Phi Karte zugeschnitten sind, mit dem Tag
\texttt{platform:knc} markiert.

Im Folgenden wird nochmals speziell auf die Moduldefinitionen eingegangen, da
diese insbesondere bei der Entwicklung von Anwendungen und Erweiterung des
Kernels zum Einsatz kommen.

\begin{lstlisting}[float, label=lst:module, caption=Ein Beispiel einer
Modulbeschreibung (\texttt{mcconf.module}).]
[module.boot-memory-multiboot]
requires = ["platform:multiboot"]
incfiles = ["boot/memory/Stage3Setup.h"]
kernelfiles = ["boot/memory/Stage3Setup.cc", "boot/memory/Stage3Setup.cc",
"boot/memory/Stage3Setup-multiboot.cc"]

[module.boot-memory-gem5]
requires = ["platform:gem5"]
incfiles = ["boot/memory/Stage3Setup.h"]
kernelfiles = ["boot/memory/Stage3Setup.cc", "boot/memory/Stage3Setup.cc",
"boot/memory/Stage3Setup-e820.cc"]

[module.boot-memory-knc]
requires = ["platform:knc"]
incfiles = ["boot/memory/Stage3Setup.h"]
kernelfiles = ["boot/memory/Stage3Setup.cc", "boot/memory/Stage3Setup.cc",
"boot/memory/Stage3Setup-sfi.cc", "boot/memory/Stage3Setup-knc.cc"]
\end{lstlisting}

Jedes Modul befindet sich mit seinen Quelldateien und der Spezifikation in einem
eigenen Ordner. Die Spezifikation wird dabei in einer oder mehreren
\texttt{*.module}-Dateien festgehalten. Die Pfade zu den einzelnen Quelldateien
sind darin relativ zum Pfad der Spezifikation (\texttt{*.module}-Datei)
angegeben (siehe Listing~\ref{lst:module}). Der Pfad zum Verzeichnis, das alle
vorhandenen Module beinhaltet, ist Teil der Konfigurations-Spezifikation.
Listing \ref{lst:module} zeigt die Spezifikation eines Moduls, das die Datei
\texttt{Stage3Setup.h} für unterschiedliche Plattformen bereitstellt. Anhand des
platform-Tags wird die korrekte Implementierungs-Variante ausgewählt.

Um ein komplettes Kernel-Image zu erstellen liest das \texttt{mcconf}-Tool alle
vorhandenen Modulbeschreibungen ein, wählt anhand einer
Konfigurations-Spezifikation Module aus um vorhandene Abhängigkeiten aufzulösen
und kopiert die ausgewählten Quell-Dateien in einen Zielordner.
Außerdem generiert das Tool ein \texttt{Makefile}, das Regeln für die
Übersetzung der Quelldateien und das Linken des fertigen Kernel-Images enthält.
Diese Regeln können, wie bereits beschrieben, innerhalb der einzelnen
Modul-Spezifikationen spezialisiert werden, die die entsprechenden Teile des
\texttt{Makefile} enthalten.

Sobald das angepasste \texttt{Makefile} für die gesamte Kernel-Konfiguration im
entsprechenden Unterverzeichnis erzeugt wurde, kann diese übersetzt und
ausgeführt werden. Dafür enthält das \texttt{Makefile} automatisch generierte
Targets wie \texttt{make micrun} -- zum Starten auf einem Knights Corner
Prozessor -- \texttt{make qemu} -- zur Emulation mit QEMU --, oder zur
Aktivierung im GDB Debug-Modus \texttt{make qemudgb}. Die Übersetzung und das
Erstellen des Kernel-Abbilds geschieht durch \texttt{make}.

Für die Konfiguration des Build-Prozesses können separate Module verwendet
werden, die keine Quelldateien enthalten. Damit können beispielsweise
je nach Zielplattform  unterschiedliche Compiler verwendet werden.
Diese Module enthalten lediglich Verweise auf unterschiedliche
Submodule und deren Versionen oder Code, der direkt in das \texttt{Makefile}
übernommen wird und damit die Übersetzung von \mythos an die Zielplatform
anpasst.

Weitere Informationen und Beispiele für die Funktionsweise des \texttt{mcconf}
Tools sowie für die Definition und Umsetzung von
Modul-Konfigurationen befinden sich im Projekt-Verzeichnis
\texttt{\{mythos\}/3rdparty/mcconf}.

\section{Debugging}
Für die Fehlersuche im Kernel und in Anwendungsprogrammen kamen zunächst
hauptsächlich klassische Debugger in Kombination mit Emulatoren wie Qemu und
Bochs zum Einsatz. Wie bereits erklärt, enthält das \texttt{Makefile} einer
jeden Kernel-Konfiguration entsprechende Targets (wie z.B. \texttt{qemugdb},
\texttt{qemu-ddd}). Der Emulator und der Debugger können über
die Remote-Debugging Werkzeuge beliebiger IDEs, wie z.B. Eclipse, KDevelop
etc., angesteuert werden, was das Debugging des Kernels im Emulator erleichtert.
Für diese Fälle ist die Kernel-Konfiguration \texttt{kernel-amd64.config}
vorgesehen.

Um jedoch die Ausführung von parallelem Code auf realer Hardware wie dem Xeon
Phi Coprozessor nachvollziehen zu können, sind zusätzliche Werkzeuge notwendig,
da die dortige Ausführung des Codes nicht beliebig unterbrochen werden kann.

Zu diesem Zweck wurde ein Tracing Werkzeug entwickelt, das verteilt über alle
Hardwarethreads Ausführungs- und Laufzeit-Informationen (Traces) sammelt und
diese aggregiert ausgeben kann.
Hierfür wird auf jedem Hardwarethread separater Speicher verwendet, um
Synchronisation beim Sammeln einzelner Traces zu vermeiden, da dadurch die
Ausführung selbst beeinflusst werden würde. Das Tracing Werkzeug wird momentan erweitert
und an die aktuelle Kernel-Architektur angepasst. Entsprechende Dokumentationen
und Anleitungen werden auf dem öffentlichen Repository
(\url{https://github.com/ManyThreads/mythos}) bereitgestellt und auf der
Projektseite (\url{https://manythreads.github.io/mythos/}) bekannt gegeben.

Zusätzlich wurde ein neues Werkzeug entwickelt, mit dem kausale Zusammenhänge
zwischen Ereignissen auf verschiedenen Hardwarethreads extrahiert und
anschließend für die Programmanalyse verwendet werden können. Solche Ereignisse
können beispielsweise durch Nachrichtenaustausch hervorgerufene Funktionsaufrufe
sein. Durch dieses Werkzeug ist es möglich, die Ausführung von parallelem Code
im Nachhinein genau zu untersuchen und wichtige Informationen über Parallelität
und Abhängigkeiten zu extrahieren. Dadurch können Deadlocks oder
Kommunikationsmuster deutlich einfacher untersucht werden. Der Code und die
entsprechenden Dokumentationen des Tools werden gerade ebenfalls portiert und
erweitert und nach einer entsprechenden Evaluation und Testphase öffentlich zur
Verfügung gestellt.

% ------------------------------------------------------------------------------
% \bibliographystyle{alpha}
% \bibliography{literature}

\end{document}
